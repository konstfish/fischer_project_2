\documentclass[12pt, letterpaper]{article}
\usepackage[utf8]{inputenc}
\usepackage{cite}
\usepackage{float}
\usepackage{tikz}
\usepackage{hyperref}
\usepackage[newfloat]{minted}
\usepackage{caption}
\usepackage{dirtree}
\tolerance=1
\emergencystretch=\maxdimen
\hyphenpenalty=10000
\hbadness=10000

\newenvironment{code}{\captionsetup{type=listing}}{}
\SetupFloatingEnvironment{listing}{name=Source Code}

\graphicspath{{images/}}

\title{Simpler POP3 Client mit TLS Support}
\author{David Fischer, 5CHIF}
\date{April 2021}

\begin{document}

\begin{titlepage}
\maketitle
\end{titlepage}

\tableofcontents
\newpage

\section{Einleitung}

\subsection{Problematik}


\section{Implementierung}

\subsection{Kurze Beschreibung diverser Codeblöcke}


\begin{code}
\begin{minted}{cpp}
// create vector of threads to store the nodes operator()() threads
vector<thread> node_container;
node_container.resize(no_of_nodes);

for (int cnt = 0; cnt < no_of_nodes; cnt++) {
    Node tmp_node(cnt, ref(coord), opt);
    node_container.at(cnt) = thread{tmp_node};
}

// join threads from node_container
for( auto &t : node_container ) {
    t.join();
}
\end{minted}
\caption{Erstellen sowie Befüllen des node\_container Vectors.}
\label{thread_vector_ref}
\end{code}


\subsection{Externe Bibliotheken}
\label{extBib}

\subsubsection{CLI11}
CLI11\cite{cli11_ref} ermöglicht eine einfache Verarbeitung von Kommandozeilenargumenten mit eingebauten Methoden zum Überprüfen der angegebenen Werte. Auf diese Argumente wird in Sektion \ref{usage} näher eingegangen.

\subsubsection{tabulate}
Sobald das Programm vom Nutzer mittels $Ctrl+C$ abgebrochen wird, wird eine Tabelle mittels tabulate\cite{tabulate-ref} erstellt. Diese enthält diverse Informationen die während der Programmlaufzeit gesammelt wurden.
Wenn die httplib Flag gesetzt ist, kann auf die Tabelle, wie in Paragraph \ref{get_ref} erwähnt, jederzeit zugegriffen werden.

\section{Verwendung}
\label{usage}

\subsection{Kommandozeilenargumente}
\subsubsection{Erforderlich}

\paragraph{number}
Nummer an Workerthread, die erstellt werden soll. Standardmäßig limitiert auf 2 bis 200. Sollte die -r Flagge gesetzt sein, ist die Limitation von 2 auf 10 reduziert.

\subsubsection{Optional}
\paragraph{-o, --outage-simulation}
Lässt Workerthreads mit einer 3 \% Chance pro Eintritt in den kritischen Abschnitt ausfallen, um einen Deadlock zu erzeugen. 

\paragraph{-d, --outage-detection}
Nur zulässig, wenn --outage-simulation auch gesetzt ist. Lässt den Koordinator 8 Sekunden auf die Node warten. Wenn nach dieser Zeit keine REL Anfrage eingeht, wird die Node aus der Queue entfernt.

\paragraph{-r, --requests}
Lässt Kommunikation zwischen den Threads über HTTP laufen. Näher beschrieben in \ref{httplib_ref}.

\newpage

\section{Projektstruktur}
\dirtree{%
  .1 /.
  .2 LICENSE.
  .2 meson\_options.txt.
  .2 meson.build.
  .2 README.md.
  .2 .gitignore.
  .2 include.
  .3 utils.h.
  .3 Node.h.
  .3 Coordinator.h.
  .2 src.
  .3 utils.cpp.
  .3 Node.cpp.
  .3 Coordinator.cpp.
  .3 main.cpp.
  .2 doc.
  .3 ausarbeitung.tex.
  .3 references.bib.
  .3 ausarbeitung.pdf.
  .2 build.
}\hfill

% .bib include & references
\newpage
\bibliography{references}
\bibliographystyle{plain}
\end{document}